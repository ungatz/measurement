\documentclass[12pt]{article}

% --- Packages ---
\usepackage[utf8]{inputenc}    % Encoding
\usepackage{amsmath, amssymb, stmaryrd, mathtools}  % Math symbols
\usepackage{graphicx}          % For figures
\usepackage{hyperref}          % Clickable links
\usepackage{geometry}          % Page layout
\usepackage{braket}            % Dirac speak
\usepackage{cancel}
\geometry{margin=1in}

\usepackage[dvipsnames]{xcolor} % gives lots of named colors
\newcommand{\gr}[1]{\textcolor{ForestGreen}{#1}}
\newcommand{\rd}[1]{\textcolor{Maroon}{#1}}

% Fonts
\usepackage{libertine}
\usepackage{libertinust1math}
\usepackage[T1]{fontenc}

\newcommand{\kw}[1]{\mathsf{#1}}

% --- Title Information ---
\title{Adding measurement to ISO}
% \author{Your Name \\ \small Your Institution}
% \date{\today}

\begin{document}

\maketitle

% \begin{abstract}
% Write a brief summary of your paper here. Keep it concise and informative.
% \end{abstract}

% \section{Introduction}
% Introduce the topic, provide background, and state the purpose of your paper.

\section{Related Work}
There are two most relevant papers that attempt to solve the same problems:
\begin{itemize}
  \item combine classical and quantum data in a syntactically ergonomic way;
  \item allow impure operations by allowing measurements;
  \item should be able to express quantum channels like bit-flip etc.
\end{itemize}

\textbf{1. Qunity, Voichick et. al.}
\vspace{0.75em}

\textit{Main Idea:} Allow not only unitaries but isometries and superoperators as trace preserving and trace
non-increasing maps respectively. Pure expressions in Qunity have the same typing rules as the reversible classical
language $\Pi$.

The new thing is the ability to express mixed states where the classical and quantum data are expressed
in a unified way using the $\textsf{try-catch}$ block.

Pure types are considered subtypes of mixed types.

\[
\frac{\Delta_0 \vdash e_0 : T \quad \Delta_1 \vdash e_1 : T}{\Delta_0, \Delta_1 \vdash \textsf{try}\, e_0\, \textsf{catch}\, e_1 : T}
\quad \text{T-Try}
\]

\[
\begin{aligned}
\llbracket \Delta_0, \Delta_1 \vdash \mathrm{try}\ e_0\ \mathrm{catch}\ e_1 : T \rrbracket
&\bigl( \ket{\tau_0,\tau_1}  \ket{\tau'_0,\tau'_1} \bigr) \notag \\
&:= \llbracket \Delta_0 \vdash e_0 : T \rrbracket
\bigl( \ket{\tau_0}  \ket{\tau'_0} \bigr) \notag \\
&\quad + \Bigl( 1 - tr\!\bigl( \llbracket \Delta_0 \vdash e_0 : T \rrbracket
  ( \ket{\tau_0}  \ket{\tau'_0} ) \bigr) \Bigr) \cdot
\llbracket \Delta_1 \vdash e_1 : T \rrbracket
\bigl( \ket{\tau_1}  \ket{\tau'_1} \bigr)
\end{aligned}
\]

Measurement is implemented in an ad-hoc way as the following function $\textsf{Bit} \Rightarrow \textsf{Bit}$ using the
concept that ``partial trace (or discarding) is equivalent to measuring and discarding''

The function first copies the input (sharing via entanglement) and throws away the copy
where copying only copies the basis element thus implementing the isometry $\ket{00}\bra{0} + \ket{11}\bra{1}$ and \textsf{fst} works as partial trace.

\[
\text{meas}_{T} := \lambda x \mapsto (x, x) \triangleright \text{fst}_{Bit \otimes Bit}
\]

An example usage of this measurement:
\[
\text{coin} := \text{meas}_{Bit} (\text{had } 0) = () \triangleright \text{left}_{Bit} \triangleright \text{had}
   \triangleright (\lambda x  (x, x) )
   \triangleright (\lambda (x_0, x_1)  x_0 )
   ^ {T  \otimes T}
\]

\textit{Remarks:}
\begin{itemize}
  \item Instead of operational semantics they provide a compilation procedure to low level circuits in \textsf{QASM}.
        \item They mention that patterns in match constructs can be non-exaustive but this does not follow from their
        typing rules.
\end{itemize}

\textbf{2. Combining quantum and classical control, Dave el. al.}
\vspace{0.75em}

\textit{Main Idea:}
\begin{itemize}
  \item a new modality to incorporate pure quantum types in a mixed quantum typesystem;
  \item modify ``quantum configurations'' of quantum lambda calculus to give operational semantics to mixed
        quantum computations;
  \item denotational semantics motivated by von Neumann algebras.
\end{itemize}

Term grammar for the pure quantum subsystem is that of ISO.

Instead of an operational semantics they give an equational theory for pure subsystem.

The classical subsystem is basically a call by value linear lambda calculus which can work with types from quantum
fragment (the bold font).

$\mathcal{B}(\textbf{Q})$ is the type that represents mixed state quantum computation on the Hilbert space determined by
the type \textbf{Q}. This $\mathcal{B}$ is the new modality where $\mathcal{B}(H)$ represents the space of bounded
operators on $H$ which is a vN algebra.

The rule for the term \textsf{pure}(t) introduces a state t from the quantum control fragment as a term of type $\mathcal{B}(\textbf{Q})$ into the main calculus. The typing rule for measurement maps a term M of quantum type $\mathcal{B}(\textbf{Q})$ to a term \textsf{meas}(M) of classical type $\bar{\textbf{Q}}$, where the type $\bar{\textbf{Q}}$ is defined inductively on the structure of the pure quantum type $\textbf{Q}$.

\[
  \frac{\Delta \vdash M : \mathcal{B}(\textbf{Q})}{\Delta \vdash \mathrm{meas}(M) : \overline{\textbf{Q}}}
  \hspace{2em}
\frac{\cdot \vdash \textbf{t} : \textbf{Q}}{!\Delta \vdash \mathrm{pure}(\textbf{t}) : \mathcal{B}(\textbf{Q})}
\]

Operationally,
\[
\begin{aligned}
(t, u_{\sigma}, \mathrm{pure}(t')) &\;\xrightarrow{1}\;
\bigl(t \otimes t',\, u_{\sigma \mathrm{swap}} \circ (u_{\sigma} \otimes u_{\mathrm{id}}),\, x\bigr) \\[1em]
(\Sigma_i p_i \cdot \Sigma_j \boldsymbol{\alpha}_{ij} \cdot \boldsymbol{b}'_{ij} \otimes \cdots \otimes \boldsymbol{b}_i \otimes \cdots,\,
u_{\sigma_s},\, \mathrm{meas}(x))
&\;\xrightarrow{|\!p_k\!|^2}\;
\bigl(\Sigma_j \boldsymbol{\alpha}_{kj} \cdot \boldsymbol{b}'_{kj} \otimes \cdots,\,
u_{\mathrm{id}},\, \overline{\boldsymbol{b}}_k \bigr)
\end{aligned}
\]

\textit{Remarks:}
\begin{itemize}
        \item The use of von Neumann algebra in denotational semantics is interesting since it relates more closesly to
        physical realizations.
  \item writing programs is not so straightforward (might be biased on this).
\end{itemize}

\section{Ideas}

\[
  \begin{array}{@{}l@{\qquad}l@{}}
    \text{Val \& term types} &
                               a,b ::= 1 \mid a \oplus b \mid a \otimes b \mid [a] \\[0.8ex]
    \text{Iso types} &
                       T ::= a \leftrightarrow b \mid \bcancel{(a \leftrightarrow b) \to T} \mid \gr{a \Rightarrow b} \\[1.2ex]
    \text{Pure values} &
                         v ::= () \mid x \mid \kw{inj}_{\ell}\, v \mid \kw{inj}_{r}\, v \mid \langle v_1, v_2\rangle \\[0.8ex]
    \text{Combination of values} &
                                   e ::= v \mid e_1 + e_2 \mid \alpha\, e \\[0.8ex]
    \text{Products} &
                      p ::= () \mid x \mid \langle p_1, p_2\rangle \\[0.8ex]
    \text{Extended Values} &
                             e ::= v \mid \kw{let}\; p_1 = \omega\; p_2\; \kw{in}\; e \\[0.8ex]
    \text{Isos} &
                  \omega ::= \{\, \mid\, v_1 \leftrightarrow e_1 \mid v_2 \leftrightarrow e_2 \ \ldots \,\}
                  \mid \lambda f.\,\omega \mid \bcancel{\mu f.\,\omega} \mid f \mid \omega_1\,\omega_2 \\[0.8ex]
    \gr{\text{CPMs}} &
                       \Lambda ::= \{\, \mid\, v_1 \leftrightarrow e_1 \mid v_2 \leftrightarrow e_2 \ \ldots \,\}
                       \mid \lambda f. \Lambda \mid f \mid \text{does } \Lambda_{1}\Lambda_{2} \text{ make sense here? }\\[0.8ex]
    \text{Terms} &
                   t ::= () \mid x \mid \kw{inj}_{\ell}\, t \mid \kw{inj}_{r}\, t \mid \langle t_1, t_2\rangle \mid
                   \omega\, t \mid \kw{let}\; p = t_1\; \kw{in}\; t_2 \mid t_1 + t_2 \mid \alpha \cdot t \mid \\  & \ \ \ \ \ \ \ \ \ \gr{\kw{meas}(t)} \mid \gr{\kw{alloc}(\rd{?})} \mid \gr{\Lambda t}
  \end{array}
\]

\textsf{meas(t)} when $t$ is pair?



Typing for an iso in this quantum extention originally is given as:

\[
\frac{%
\begin{array}{l@{\qquad\qquad}l}
\Delta_{1};\,\Psi \vdash_{v} v_{1} : a & \cdots \quad \Delta_{n};\,\Psi \vdash_{v} v_{n} : a \\[0.3ex]
\Delta_{1};\,\Psi \vdash_{v} e_{1} : b & \cdots \quad \Delta_{n};\,\Psi \vdash_{v} e_{n} : b \\[0.6ex]
\kw{OD}_{a}\{v_{1},\ldots,v_{n}\} & \kw{OD}_{b}^{\mathrm{ext}}\{e_{1},\ldots,e_{n}\} \\[1ex]
\multicolumn{2}{c}{%
\left(
\begin{array}{ccc}
a_{11} & \cdots & a_{1n} \\
\vdots & \ddots & \vdots \\
a_{n1} & \cdots & \bcancel{a_{nn}}\ \gr{a_{nm}}
\end{array}
\right)\ \text{is } \bcancel{\text{unitary}} \text{ isometry}%
}
\end{array}
}{%
\Psi \vdash_{\omega}
\left\{
\begin{array}{l}
v_{1} \leftrightarrow a_{11}\cdot e_{1} + \cdots + a_{1n}\cdot e_{n} \\
\vdots \\
v_{n} \leftrightarrow a_{n1}\cdot e_{1} + \cdots + a_{nn}\cdot e_{n}
\end{array}
\right\}
:\ \bcancel{a \leftrightarrow b}\ \gr{a \Rightarrow b}
}
\]
\end{document}



% \section{Conclusion}

\bibliographystyle{plain}
\bibliography{references}

\end{document}
